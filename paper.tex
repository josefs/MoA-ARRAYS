%-----------------------------------------------------------------------------
%
%               Template for sigplanconf LaTeX Class
%
% Name:         sigplanconf-template.tex
%
% Purpose:      A template for sigplanconf.cls, which is a LaTeX 2e class
%               file for SIGPLAN conference proceedings.
%
% Guide:        Refer to "Author's Guide to the ACM SIGPLAN Class,"
%               sigplanconf-guide.pdf
%
% Author:       Paul C. Anagnostopoulos
%               Windfall Software
%               978 371-2316
%               paul@windfall.com
%
% Created:      15 February 2005
%
%-----------------------------------------------------------------------------


\documentclass{sigplanconf}

% The following \documentclass options may be useful:

% preprint      Remove this option only once the paper is in final form.
% 10pt          To set in 10-point type instead of 9-point.
% 11pt          To set in 11-point type instead of 9-point.
% authoryear    To obtain author/year citation style instead of numeric.

\usepackage{amsmath}
\usepackage{longtable}
\usepackage{stmaryrd}
\usepackage[mathletters]{ucs}
\usepackage[utf8x]{inputenc}

\begin{document}

\special{papersize=8.5in,11in}
\setlength{\pdfpageheight}{\paperheight}
\setlength{\pdfpagewidth}{\paperwidth}

\conferenceinfo{CONF 'yy}{Month d--d, 20yy, City, ST, Country} 
\copyrightyear{20yy} 
\copyrightdata{978-1-nnnn-nnnn-n/yy/mm} 
\doi{nnnnnnn.nnnnnnn}

% Uncomment one of the following two, if you are not going for the 
% traditional copyright transfer agreement.

%\exclusivelicense                % ACM gets exclusive license to publish, 
                                  % you retain copyright

%\permissiontopublish             % ACM gets nonexclusive license to publish
                                  % (paid open-access papers, 
                                  % short abstracts)

\titlebanner{banner above paper title}        % These are ignored unless
\preprintfooter{short description of paper}   % 'preprint' option specified.

\title{Title Text}
\subtitle{Subtitle Text, if any}

\authorinfo{Josef Svenningsson}
           {Department of Computer Science of Engineering\\Chalmers University of Technology}
           {josef.svenningsson@chalmers.se}
\authorinfo{Lenore M. Mullin}
           {Emeritus Professor\\Department of Computer Science\\
University at Albany, SUNY}
           {lenore@cs.albany.edu}

\maketitle

\begin{abstract}
This is the text of the abstract.
\end{abstract}

\category{CR-number}{subcategory}{third-level}

% general terms are not compulsory anymore, 
% you may leave them out
\terms
term1, term2

\keywords
keyword1, keyword2

\section{Introduction}

Computing with arrays has been a constantly important paradigm in the
history of computing. For this reason we have seen a plethora of
formalisms and languages being developed to aid theoriticians and
developers to produce more suitable abstractions and faster programs.
Examples include Fortran, APL and Matlab but the list goes on and on.
Having many different options to choose from can be a good thing but
it can also lead to a balkanization of the community. When researcher
dedicated to different formalisms can no longer talk to each other it
inhibits collaborations and limits the impact of innovatation.

In this paper we aim to rectify some of the fragmentation of the
high-performance array programming community by relating two
particular formalisms: Mathematics of Arrays and Pull arrays. On the
surface they appear rather different but at their core they share many
principles and properties which we will examine in this paper. In
particular we will highlight the following points:
\begin{itemize}
\item We give an introduction to both MoA (section \ref{sec:moa}) and Pull
  arrays (section \ref{sec:pull}).
\item The two formalisms are compared with respect to types (section
  \ref{sec:types}), fusion/\(\psi\)-reduction (section
  \ref{sec:normalization}), how they deal with higher dimensions
  (section \ref{sec:highdim}) and notation (section \ref{sec:notation}).
\end{itemize}

We would like to stress that we are not looking to declare a
winner. Instead our intention is that this paper can serve as a help
for the two communities to be able to learn about each others work.

\section{Overview of MoA}
\label{sec:moa}

% See the ``Article customise'' template for come common customisations
\newcommand{\cat}{+\!\!\!+}
%\newcommand{\cat}{\mbox{$+\!\!+_s$}}
\newcommand{\Take}{\mbox{Take}}
\newcommand{\take}{\,\bigtriangleup\,}
\newcommand{\Drop}{\mbox{Drop}}
\newcommand{\drop}{\,\bigtriangledown\,}
\newcommand{\rshp}{\,\widehat{\Rho}\;}
\newcommand{\Rho}{\,\rho\,}
\newcommand{\Tau}{\,\tau\,}
\newcommand{\Dim}{\,\delta\,}
\newcommand{\transpose}{\bigcirc\!\!\!\!\!\backslash\;}
\newcommand{\kron}{\bigcirc\,\!\!\!\!\!\!\times\;}
\newcommand{\Ravel}{\,\mbox{\tt rav}\,}
\newcommand{\Gradeup}{\,\mbox{\tt gu}\,}
%\newcommand{\cat}{+\!\!\!\!+}

\newtheorem{definition}{Definition}
%\newtheorem{definition}{Definition}
\newtheorem{corollary}{Corollary}
\newtheorem{theorem}{Theorem}

\noindent

MoA, an acoronym for {\em A Mathematics of Arrays\cite{thesis}}, is a theory of arrays with both an 
algebra and a calculus of 
indexing based on shapes, i.e. sizes of each dimension. The index calculus, refered to as {\em The Psi Calculus}, 
gets its names from the {\em Psi}  function which is the foundational function in the theory. These results began with the awareness that Iverson's APL\cite{iverson62}
was inspired by Sylvester's Universal Algebra\cite{sylvester} and the fact that he and Cayley invented matrices.

MoA and the Psi Calculus, as a theory,  have  also been used to describe and verify 
hardware\cite{hardy1,hardy2,fftharry}, describe the partitioning and
scheduling of array operations (and their costs)\cite{mdst93,lee95,aachen96,coffin94,ll2,mcmahon95,haleh98}, 
and the formulation of parallel and distributed processing\cite{Mul03}. 
MoA and $\psi$-Calculus, can be used to abstract complex processor memory layouts 
for tensor/array expressions, reduce these expressions first to a semantic 
normal form (Denotational Normal Form-DNF) then to an Operational Normal Form 
(ONF). 
%
%
%
%\bibliographystyle{plain}
%\bibliography{paper1,hpf,workshop}
\subsection*{Elements of the theory}
\subsection*{\label{indshp} Indexing and Shapes}

The central operation of MoA is the indexing function
\[
                    p \psi A \]
in which a vector of $n$ integers $p$ is used to select an item of 
the $n$-dimensional array $A$.
The operation is generalized to select a partition of $A$, 
so that if $q$ has only $k < n $ components
then
\[                    q \psi A \]
is an array of dimensionality $n-k$ and $q$ selects among the possible 
choices for the first $k$ axes. In MoA zero origin indexing is assumed.
For example, if $A$ is the $3$ by $5$ by $4$ array
\small
\[ \left [
\begin{array}{lrrrr}
0 & 1 &  2 & 3  \\
4 & 5 & 6 & 7 \\
8 & 9 & 10 & 11  \\
12 & 13 & 14 & 15 \\
16 & 17 & 18 & 19 
\end{array} \right ] 
\left [ \begin{array}{rrrr}
20 & 21 & 22 & 23 \\
24 & 25 & 26 & 27 \\
28 & 29 & 30 & 31 \\
32 & 33 & 34 & 35 \\
36 & 37 & 38 & 39 
\end{array} \right ]
\left [ \begin{array}{rrrr}
40 & 41 & 42 & 43 \\
44 & 45 & 46 & 47 \\
48 & 49 & 50 & 51 \\
52 & 53 & 54 & 55 \\
56 & 57 & 58 & 59
\end{array} \right ] \]
\normalsize
then
\[
     <1> \psi A = \left [  \begin{array}{rrrr}
                   20 & 21 & 22 & 23 \\
                   24 & 25 & 26 & 27 \\
                   28 & 29 & 30 & 31 \\
                   32 & 33 & 34 & 35 \\
                   36 & 37 & 38 & 39 
                    \end{array} \right ] \]

\[     <2 \; 1> \psi A \; \; = \; \; < \; 44 \;\; 45 \;\; 46 \;\; 47 \; > \]

\[      <2\; 1\; 3> \psi A \; \; = \; \; 47 \]



	We now introduce notation to permit us to define $\psi$ formally. We will use $A, B, ...$ to denote an array of numbers (integer, real, or complex).
An array's dimensionality will be denoted by $d_A$ and will be assumed to 
be $n$ if not specified.

	The shape of an array $A$, denoted by $s_A$, is a vector of integers of length $d_A$,
each item giving the length of the corresponding axis. The total number of items in an
array, denoted by $t_A$, is equal to the product of the items of the shape. The subscripts
will be omitted in contexts where the meaning is obvious.

A full index is a vector of $n$ integers that describes one position 
in an $n$-dimensional
array. Each item of a full index for $A$ is less than the corresponding item of
$s_A$. There are precisely $t_A$ indices for an array(due to a
zero index origin). A partial index of $A$ is a vector of $0 \leq  k < n $
integers with each item less than the corresponding item of $s_A$.

	We will use a tuple notation (omitting commas) to describe vectors of a fixed length.
For example,
\[                  <i \; j \; k> \]
denotes a vector of length three. $<>$ will denote the empty vector.

	For every $n$-dimensional array $A$, there is a vector of the items of $A$,
which we denote by the corresponding lower case letter, here $a$. The length of the vector of
items is $t_A$. A vector is itself a one-dimensional array, whose shape is the one-item
vector holding the length. Thus, for $a$, the vector of items of $A$, the
shape of $a$ is
\[                     s_a =  \; < t_A> \]
and the number of items or total number of components
$a$\footnote{We also use $\tau a , \delta a,$ and $\rho a$ to denote
total number of components, dimensionality and shape of a. } is
\[                    t_a =  t_A . \]

	The precise mapping of $A$ to $a$ is determined by a one-to-one 
ordering function, $gamma$
\footnote{There are a family of gamma functions: 
$\gamma_{row}, \gamma_{column}, \gamma_{regular sparse}$, etc. In this case we assume row major ordering.}.  
Scalars are introduced as arrays with an empty shape vector.

There are two equivalent ways of describing an array $A$:
\begin{description}
\item[(1)] by its shape and the vector of items, i.e.  $A = \{s_A , a\}$, or
\item[(2)] by its shape and a function that defines the value at every index $p$.
\end{description}
These two forms have been shown to be formally equivalent~\cite{jenk94}.
We wish to use the second form in defining functions on multidimensional
arrays using their Cartesian coordinates (indices). 
The first form is used in describing
address manipulations to achieve effective computation.

To complete our notational conventions, we assume that $p,q,...$ will 
be used to denote
indices or partial indices and that $u,v,..$ will be used to denote 
arbitrary vectors of integers.
In order to describe the $i_{th}$ item of a vector $a$, either $a_i$ or 
$a[i]$ will be used. If $u$ is a
vector of $k$ integers all less than $t_A$, then $a[u]$
will denote the vector of length $k$, whose items are the items of $a$ at 
positions $u_j$, $j = 0,...,k-1.$

Before presenting the formal definition of the $\psi$ indexing function we 
define a few functions on vectors:
\begin{tabbing}

\hspace{.25cm}\= $u \;\cat\; v$ \hspace{1cm} \=		\=catentation of vectors $u$ and $v$ \\
\>$u$ + $v$\>		\>itemwise vector addition assuming $t_u = t_v$ \\
\>$u \times v$ \>	\>	itemwise vector multiplication \\
\>$n$ + $u$, $u$ + $n$ \>	addition of a scalar to each item of a vector \\
\>$n \times u$, $u \times n$ \>   multiplication of each item of a vector by a scalar \\
\>$\iota \; n$ \>\>  the vector of the first n integers starting from $0$ \\
%\footnote{$\iota \vec n$ is also defined and  produces an array of 
%cartesian coordinates~\cite{mul88}.}\\
\>$\pi \; v$ \> \>a scalar which is the product of the components of v \\
\>$k \;\take \;u$ \>\>	when $k \geq 0$ the vector of the first $k$ items of $u$) \\
\> \> \>and when $k<0$ the vector of the last $k$ items of $u$ \\
\>$k \;\drop \;u$ \>  \> when $k \geq 0$ the vector of $t_u - k$ last items of $u$) \\
\> \> \>and when $k<0$ the vector of the first $t_u - |k|$ items of $u$ \\
\end{tabbing}

\begin{definition}
Let A be an n-dimensional array and p a vector of integers.
If p is an index of A, 
\[             p \psi A = a[\gamma(s_A,p)], \]
   where 
\begin{eqnarray}
      \gamma(s_A,p) & = & x_{n-1}  \;\;\;\;\; \mbox{defined by 
the recurrence}    \nonumber \\ 
                 x_0 & = & p_0, \nonumber \\ 
              x_j & = & x_{j-1} * s_j + p_j, \; \; \; j=1,...,n-1. \nonumber
\end{eqnarray}

If $p$ is partial index of length $k < n,$
\[             p \psi A = B \]
where the shape of $B$ is
\[         s_B = k \drop s_A,    \] 
and for every index $q$ of $B$,
\[         q \psi B = (p \cat q) \psi A \]
\end{definition}
The definition uses the second form of specifying an array to define the result
of a partial index. For the index case, the function $\gamma(s,p)$ 
is used to convert an index $p$ to an
integer giving the location of the corresponding item 
in the row major order list of items of an array of shape $s$. The
recurrence computation for $\gamma$ is the one used in most compilers 
for converting an
index to a memory address~\cite{djr}.

\begin{corollary}
$<> \psi$ A = A.
\end{corollary}

The following theorem shows that a $\psi$ selection with a partial index can
be expressed as a composition of $\psi$ selections.

\begin{theorem}
Let A be an n-dimensional array and p a partial index so 
that $p = q \cat r.$ Then
\[                 p \psi A = r \psi (q \psi A) . \]

Proof: The proof is a consequence of the fact that for vectors u,v,w
\[           (u \cat v) \cat w = u \cat (v \cat w). \]
If we extend p to a full index by $p \cat p',$ then

\begin{eqnarray}
             p' \psi (p \psi A) & = & (p \cat p') \psi A \nonumber \\
                             & = & ((q \cat r) \cat p') \psi A \nonumber \\
                             &  =  & (q \cat (r \cat p')) \psi A \nonumber \\
                            &  =  & (r \cat p') \psi (q \psi A) \nonumber \\
                            &   = & p' \psi ( r \psi (q \psi A)) \nonumber \\
                          p \psi A & = & r \psi (q \psi A) \nonumber                   
\end{eqnarray}

\end{theorem}
{\em which completes the proof.}

We can now use {\em psi} to define other operations on arrays. 
For example, consider definitions of {\em take} and {\em drop} for multidimensional arrays.
\begin{definition}[take: $\take$]
Let A be an n-dimensional array, and k a non-negative integer
such that $0 \leq k < s_0$. Then
\[            k \take A = B \]
where
\[            s_B = <k> \cat (1 \drop s_A) \]
and for every index p of B,
\[            p \psi B = p \psi A . \]

\end{definition}
(In MoA $\take$ is also defined for negative integers and is generalized to any vector u with its
absolute value vector a partial index of A. The details are omitted here.)
\begin{definition}[reverse: $\Phi$]
 Let A be an n-dimensional array. Then
\[             s_{\Phi A} = s_A \]
and for every integer i, $0 \leq i < s_0, $
\[             <i> \psi \Phi A = <s_0 - i - 1> \psi A. \]
\end{definition}
This definition of $\Phi$ does a reversal of the 0th axis of A. 



\subsection*{Higher Order Operations}

Thus far operation on arrays, such as concatenation, rotation, etc., have been 
performed over their 0th dimensions.  We introduce the higher order binary 
operation $\Omega$~\cite{mul88}  which extends operations over all dimensions.
$\Omega$ is defined when its left argument is a unary or binary
operation and its right argument is a vector describing the dimension upon 
which operations are to be performed, or which sub-arrays are used in 
operations.  The dimension upon which operations are to be performed is often 
called the {\it axis} of operation.  The result of $\Omega$ is a unary or 
binary operation.

\subsection*{From Semantic to Operational Forms}

%Section with Psi  Correspondence Theorem and thinking in arrays *\

Consider the expression

\begin{eqnarray}
X & = & (2 \take ( \Phi A)) \times ( 1 \drop (\Phi A)) 
\end{eqnarray}
That is, take the first two planes after reversing A along the primary axis then multiply that array by the last two 
planes after reversing A. We want to concurrently perform both 5 by 4 multiplications. 
A is the array previously defined with shape $<3\;5\;4>$.
Suppose also that A is in shared memory and accessible to 2 processors.
First, we {\em Psi-Reduce} our expression to its Denotational Normal Form (DNF) or semantic 
normal form. This normal form can be used to prove the equivalence of  two array expressions.

Get shape: 
\begin{eqnarray}
\rho X & = & \rho (2 \take (\Phi A)) \nonumber \\
& = & 2 \cat ((1\drop (\rho ( \Phi A)))) \nonumber \\
&=& 2 \cat ((1 \drop (\rho  A))) \nonumber \\
&=& 2 \cat <\;5\;4> \nonumber \\
&=& <2\;5\;4>
\end{eqnarray}
Now that we have the {\em shape}, we can {Psi-Reduce}.
\[ \forall \;\;\; i,j,k \;\;\; \ni \;\;\; 0 \;\;\;\leq i,j,k < \;\;\;<2\;5\;4> \]
\[<i\;j\;k> \psi X \]
\begin{eqnarray}
& =&<i\;j\;k> \psi (2 \take ( \Phi A)) \times ( 1 \drop (\Phi A)) \nonumber \\
&=&<j\;k> \psi (<i> \psi (2 \take ( \Phi A)) \times ( 1 \drop (\Phi A)) )\nonumber \\
&= &<j\;k> \psi ((<i> \psi ( 2 \take (\Phi A))) \nonumber \\
&&\;\;\;\;\;\times ( <i> \psi ( 1 \drop (\Phi A)))) \nonumber \\
&= &<j\;k> \psi (( <i> \psi (\Phi A)) \times (<i+1> \psi (\Phi A))) \nonumber \\
&=&<j\;k> \psi ((<((\rho A)[0] - 1 - i)> \psi A ) \nonumber \\
&& \;\;\;\;\;\times  (<((\rho A)[0] - 1 -( i + 1))> \psi A ) )\nonumber \\
&=&<j\;k> \psi ((< 2-i> \psi A ) \times (<1-i>  \psi A) ) \nonumber \\
&=& (< (2-i)\; j\;k> \psi A )  \times (<(i-i) \; j\;k> \psi A ) 
 \end{eqnarray}
\normalsize
This is the DNF, or semantic normal form. To proceed further we must know more about the data, i.e. layout, and architecture. 
\subsection* {Applying the Psi Correspondence Theorem: PCT}
We now want to {\em transform} the DNF above to a {\em Generic} Operational Normal Form (ONF) or way 
to {\em build the code}.  Recall that $\gamma$ maps a {\em full} index , $\vec i$, of an arbitrary array $A$ to an 
offset from the start of A, $@A$ in memory,  denoted by $a[\gamma(\vec i;(\rho A))]$.
The PCT~\cite{muljen94},  generalizes the mapping function 
gamma, $\gamma$,   and turns {\em short} indices into start, stops, and strides.
Note that we still use {\em Psi-Reduction} but now just on vectors.  
We start by analyzing the shape of the result in conjunction with the indicies involved
in the computation and observe that the indices for $j$ and $k$ are contiguous allowing
us to only use the primary axis index, a short index, in the mapping.
Thus, let Y denote $(<2-i> \psi A ) \times ( <1-i> \psi A)$. Then the shape of Y, i.e.
\begin{eqnarray}
\rho Y= 1 \drop (\rho A) = 1 \drop <3\;5\;4> = <5\;4>
\end{eqnarray}
i.e. the bounds for $j$ and $k$  above.
Continuing with the PCT to create the {\em generic} ONF on p=2 processors, we use the index of the primary axis, i, $\ni\;\;0 \leq  i < p$, since $\rho X = <2\;5\;4>$.
Using $\gamma_{row}$ for row major ordering, Y is transformed into:
\small
\begin{eqnarray}
a[ (( \gamma(<2-i >; <1 \take (\rho A>))) \times \pi ( 1 \drop (\rho A))   )  + \iota \pi 1 \drop \rho A  ] \nonumber \\
\times a[ (( \gamma(<1-i >; <1 \take (\rho A>))) \times \pi ( 1 \drop (\rho A))   )  + \iota \pi 1 \drop \rho A  ] \nonumber 
\end{eqnarray}
Reducing, we get
\begin{eqnarray}
a[ (( \gamma(<2-i >; <3>))) \times \pi ( <5\;4>)  + \iota \pi <5\;4> ] \nonumber \\
\times a[ (( \gamma(<1-i >; <3>))) \times \pi ( <5\;4>)   )  + \iota \pi <5\;4>  ]\nonumber \\
=\;\;a[( (2-i) \times  20)  + \iota 20 ] \nonumber \\
\times a[( (i-1) \times 20 )  + \iota 20 ]
\end{eqnarray}
\normalsize
which transforms the indexing into start, stride, and stop, a universal hardware description and easiliy transformed
to the mnemonics used at any hardware level ~\cite{fftharry}. 

The  {\em iota} operation depicts  20 sequential 
accesses. If we wanted to map these operations to other than  scalar registers
we could conceptually add more dimensions to analyze prefetching, buffer sizes, vector registers, caches, etc.
Suppose in this example we have two vector registers to do the addition, both with length 4, we could
abstract this operation into a 5 by 4 where we can now analyze the cost of 5 vector register loads and
stores and determine if that is more cost effective than 20 scalar operations. Thus prior to code generation we
{\em could} analyze which menomnic to map to from the ONF. 


\section{Overview of Pull arrays}
\label{sec:pull}

The functional programming community has recently gained an increased
interest in high performance parallel array programming
\cite{keller2010regular,Axelsson:2010:Feldspar,Mainland:2010:Nikola,Svensson:2011:Obsidian,Claessen:2012:Expressive,Ankner:2013:AnEDSL,lippmeier2011efficient}. One
of the central abstractions in this line of work is the \emph{pull
  array}, also know as delayed array. The can be defined as follows
(we use Haskell \cite{marlow2010haskell} to demonstrate functions
programs).

\begin{verbatim}
data Pull a = Pull { ixf    :: (Int -> a)
                   , length :: Int
                   }
\end{verbatim}

Arrays are represented as functions from index to element. We refer to
this function as an \emph{index function}. Additionally arrays also
has a length.

This representation has several advantages:
\begin{itemize}
\item \emph{Parallelism}. Since each element is computed
  independently, they can easily be computed in parallel.
\item \emph{Fusion}. When functions on pull arrays are composed, the intermediate
  arrays can be automatically removed. 
\item \emph{Compositionality}. Pull arrays provide a wealth of
  compositional and highly reusable combinators. These can be composed
  together to write high-level programs which typically are easier to
  understand than monolithic array processing code.
\end{itemize}

\begin{figure}
\begin{verbatim}
-- Mapping a function over an array
map :: (a -> b) -> Pull a -> Pull b
map f (Pull ixf l) = Pull (f . ixf) l

-- Apply a binary operation itemwise on two arrays
zipWith :: (a -> b -> c)
        -> Pull a -> Pull b -> Pull c
zipWith f (Pull ixf1 l1) (Pull ixf2 l2)
  = Pull (\i -> f (ixf1 i) (ixf2 i)) (min l1 l2)

-- Takes the n first elements of an array
take :: Int -> Pull a -> Pull a
take n (Pull ixf l) = Pull ixf (min n l)

-- Drops the n first elements of an array
drop :: Int -> Pull a -> Pull a
drop n (Pull ixf l)
  = Pull (\i -> ixf (i + n)) (max (l - n) 0)

-- Rotate an array k steps to the left
rotate :: Int -> Pull a -> Pull a
rotate k (Pull ixf l)
  = Pull (\i -> ixf ((i + k) `mod` l)) l
\end{verbatim}
\caption{Example functions on pull arrays}
\end{figure}

\subsection{Fusion}

As mentioned, one of the most important advantages of Pull arrays is
that they support
fusion\cite{gill1993short,axelsson2010feldspar,keller2010regular}. Fusion
is a program transformation which removes intermediate data structures
and is a decendant of deforestation
\cite{wadler1990deforestation}. Fusion supercedes transformations such
as loop fusion in optimizing compilers for imperative languages.  Pull
arrays make it particularly simple to achieve fusion. It simply
amounts to inlining the definitions of functions.

As an example of fusion consider the example
\verb!map (*2) (map (*5) arr)!, where each element of the array
\verb!arr! is first multiplied by five and then again by
two. Conceptually, there is an intermediate array created after the
first \verb!map! but thanks to fusion it will be removed. Fusion will proceed
as follows:

\begin{verbatim}
map (*2) (map (*5) arr)
=> { arr }
map (*2) (map (*5) (Pull ixf l)
=> { map }
map (*2) (Pull ((*5) . ixf) l)
=> { map }
Pull ((*2) . (*5) . ixf) l)
\end{verbatim}

The result is a single Pull array, the intermediate array has been removed.

The correctness of fusion relies crucially on the language being
\emph{pure}. If side effects where allowed in the indexing function
fusion would not be correct.

There are two different ways of implementing Pull arrays in the
literature. The simplest one comes from using Pull arrays in an
embedded langauge, i.e. when the array language is implemented as a
library in another host language. Examples of such languages are
Pan\cite{elliott2003compiling}, Feldspar\cite{Axelsson:2010:Feldspar},
Obsidian\cite{Svensson:2011:Obsidian} and
Nikola\cite{Mainland:2010:Nikola}. By using the technique of
implementing Pull arrays as \emph{shallow embeddings} on top of a
\emph{deeply embedded} code language, fusion comes completely for free
\cite{svenningsson2013combining}.

In the array library Repa\cite{keller2010regular}, compiler supported
rewrite rules are used to implement fusion \cite{jones2001playing}.

\subsection{Higher dimensions}

The first uses of Pull arrays used a fixed number of dimensions,
typically either one or two. However, the Repa library introduced a
new expressive notion of higher dimensionality for Push arrays
\cite{keller2010regular}. The number dimensions of an array, referred
to as the \emph{shape}, are determined statically in the type
system. The advantage of having the shape as part of the type is that
any program which produces arrays with ill-defined shapes can be ruled
out. A potential downside is that enforcing shapes in the type system might
be overly constraining and exclude useful programs. A key innovation in
Repa is the notion on \emph{shape polymorphism} which makes it
possible to write functions which accepts arrays of any shape. Below
shows the definition of the type of shapes. It is a variation on how
it is represented in Feldspar.

\begin{verbatim}
data Z
data i :. sh

data Shape sh where
  Z :: Shape Z
  (:.) :: Shape tail -> Int -> Shape (tail :. Int)
\end{verbatim}

The first two data declarations introduces new types, \verb!Z! and
\verb!:.!, without any constructors. They are used to create
snoc-lists on the type level, the definition of \verb!Shape!.  The
type \verb!Shape! is a recursive data type with two constructors. The
first constructor \verb!Z! indicates that the shape has zero
dimensions.  The second constructor \verb!:.! adds one dimension to
another shape, and stores how many elements there are in that
dimension. The type \verb!Shape! has a type parameter which is a snoc
list with one element for each dimension.

\begin{verbatim}
data Pull sh a = Pull (Shape sh -> a) (Shape sh)

map :: (a -> b) -> Pull sh a -> Pull sh a
map f (Pull ixf sh) = Pull (f . ixf) sh

zipWith :: (a -> b -> b) -> 
           Pull sh a -> Pull sh b -> Pull sh c
zipWith f (Pull ixf1 sh1) (Pull ixf2 sh2)
  = Pull (\sh -> f (ixf1 sh) (ixf2 sh2))
\end{verbatim}


\section{Comparison}

\subsection{Types}
\label{sec:types}

\subsection{\(\psi\)-reductions and fusion}
\label{sec:normalization}

\subsection{Higher dimensions}
\label{sec:highdim}

\subsection{Functions}
\label{sec:notation}

\begin{tabular}{|@{}l|l@{}|}
\hline
MoA & Feldspar
\\
\hline
δ & dim . extent
\\
ρ & extent
\\
ιξⁿ & -- generalized enumFromTo
\\
ψ & index
\\
rav & -- flattens to a one-dimensional array
\\
γ & toIndex
\\
γ' & fromIndex
\\
$s \: \hat{ρ} \: ξ$ & reshape
\\
π x & product
\\
τ & size
\\
++ & ++
\\
ξ₁ f ξ₂ & zipWith f ξ₁ ξ₁
\\
σ f ξ & map f ξ
\\
ξ f σ & map f ξ
\\
$\bigtriangleup$ & take
\\
$\bigtriangledown$ & drop
\\
$op^{red}$ & fold
\\
Φ & reverse
\\
Θ & rotate
\\
$\varobslash$ & -- generalized transpose
\\
$f Ω_d ξ$ & -- Apply $f$ along the dimension $d$ of $ξ$
\\
\hline
\end{tabular}

\subsection{Example programs}

\section{Related work}

\section{Future work}

\section{Conclusion}

\acks

Acknowledgments.

% We recommend abbrvnat bibliography style.

\bibliographystyle{abbrvnat}
\bibliography{bibliography,hpf,paper1,workshop}

% The bibliography should be embedded for final submission.

%% \begin{thebibliography}{}
%% \softraggedright

%% \bibitem[Smith et~al.(2009)Smith, Jones]{smith02}
%% P. Q. Smith, and X. Y. Jones. ...reference text...

%% \end{thebibliography}


\end{document}
