%-----------------------------------------------------------------------------
%
%               Template for sigplanconf LaTeX Class
%
% Name:         sigplanconf-template.tex
%
% Purpose:      A template for sigplanconf.cls, which is a LaTeX 2e class
%               file for SIGPLAN conference proceedings.
%
% Guide:        Refer to "Author's Guide to the ACM SIGPLAN Class,"
%               sigplanconf-guide.pdf
%
% Author:       Paul C. Anagnostopoulos
%               Windfall Software
%               978 371-2316
%               paul@windfall.com
%
% Created:      15 February 2005
%
%-----------------------------------------------------------------------------


\documentclass{sigplanconf}

% The following \documentclass options may be useful:

% preprint      Remove this option only once the paper is in final form.
% 10pt          To set in 10-point type instead of 9-point.
% 11pt          To set in 11-point type instead of 9-point.
% authoryear    To obtain author/year citation style instead of numeric.

\usepackage{amsmath}
\usepackage{longtable}
\usepackage{stmaryrd}
\usepackage[mathletters]{ucs}
\usepackage[utf8x]{inputenc}

\begin{document}

\special{papersize=8.5in,11in}
\setlength{\pdfpageheight}{\paperheight}
\setlength{\pdfpagewidth}{\paperwidth}

\conferenceinfo{CONF 'yy}{Month d--d, 20yy, City, ST, Country} 
\copyrightyear{20yy} 
\copyrightdata{978-1-nnnn-nnnn-n/yy/mm} 
\doi{nnnnnnn.nnnnnnn}

% Uncomment one of the following two, if you are not going for the 
% traditional copyright transfer agreement.

%\exclusivelicense                % ACM gets exclusive license to publish, 
                                  % you retain copyright

%\permissiontopublish             % ACM gets nonexclusive license to publish
                                  % (paid open-access papers, 
                                  % short abstracts)

\titlebanner{banner above paper title}        % These are ignored unless
\preprintfooter{short description of paper}   % 'preprint' option specified.

\title{Title Text}
\subtitle{Subtitle Text, if any}

\authorinfo{Name1}
           {Affiliation1}
           {Email1}
\authorinfo{Name2\and Name3}
           {Affiliation2/3}
           {Email2/3}

\maketitle

\begin{abstract}
This is the text of the abstract.
\end{abstract}

\category{CR-number}{subcategory}{third-level}

% general terms are not compulsory anymore, 
% you may leave them out
\terms
term1, term2

\keywords
keyword1, keyword2

\section{Introduction}

\section{Overview of MoA}

\section{Overview of Pull arrays}

The functional programming community has recently gained an increased
interest in high performance parallel array programming
\cite{keller2010regular,Axelsson:2010:Feldspar,Mainland:2010:Nikola,Svensson:2011:Obsidian,Claessen:2012:Expressive,Ankner:2013:AnEDSL,lippmeier2011efficient}. One
of the central abstractions in this line of work is the \emph{pull
  array}, also know as delayed array. The can be defined as follows
(we use Haskell \cite{marlow2010haskell} to demonstrate functions
programs).

\begin{verbatim}
data Pull a = Pull { ixf    :: (Int -> a)
                   , length :: Int
                   }
\end{verbatim}

Arrays are represented as functions from index to element. We refer to
this function as an \emph{index function}. Additionally arrays also
has a length.

This representation has several advantages:
\begin{itemize}
\item \emph{Parallelism}. Since each element is computed
  independently, they can easily be computed in parallel.
\item \emph{Fusion}. When functions on pull arrays are composed, the intermediate
  arrays can be automatically removed. 
\item \emph{Compositionality}. Pull arrays provide a wealth of
  compositional and highly reusable combinators. These can be composed
  together to write high-level programs which typically are easier to
  understand than monolithic array processing code.
\end{itemize}

\begin{figure}
\begin{verbatim}
map :: (a -> b) -> Pull a -> Pull b
map f (Pull ixf l) = Pull (f . ixf) l
\end{verbatim}
\caption{Example functions on pull arrays}
\end{figure}

\subsection{Fusion}

\subsection{Higher dimensions}

\section{Comparison}

\subsection{Types}

\subsection{\(\psi\) reductions and fusion}

\subsection{Higher dimensions}

\subsection{Functions}

\begin{tabular}{|@{}l|l@{}|}
\hline
MoA & Feldspar
\\
\hline
δ & dim . extent
\\
ρ & extent
\\
ιξⁿ & -- generalized enumFromTo
\\
ψ & index
\\
rav & -- flattens to a one-dimensional array
\\
γ & toIndex
\\
γ' & fromIndex
\\
$s \: \hat{ρ} \: ξ$ & reshape
\\
π x & product
\\
τ & size
\\
++ & ++
\\
ξ₁ f ξ₂ & zipWith f ξ₁ ξ₁
\\
σ f ξ & map f ξ
\\
ξ f σ & map f ξ
\\
$\bigtriangleup$ & take
\\
$\bigtriangledown$ & drop
\\
$op^{red}$ & fold
\\
Φ & reverse
\\
Θ & rotate
\\
$\varobslash$ & -- generalized transpose
\\
$f Ω_d ξ$ & -- Apply $f$ along the dimension $d$ of $ξ$
\\
\hline
\end{tabular}

\subsection{Example programs}

\section{Related work}

\section{Future work}

\section{Conclusion}

\acks

Acknowledgments.

% We recommend abbrvnat bibliography style.

\bibliographystyle{abbrvnat}
\bibliography{bibliography}
\bibliography{bibliography,hpf,paper1,workshop}

% The bibliography should be embedded for final submission.

%% \begin{thebibliography}{}
%% \softraggedright

%% \bibitem[Smith et~al.(2009)Smith, Jones]{smith02}
%% P. Q. Smith, and X. Y. Jones. ...reference text...

%% \end{thebibliography}


\end{document}
